\cvsection{Projects}

\begin{cventries}
  \cventry
    {} % subtitle
    {Harmony} % project
    {} % empty location
    {} % empty date
    {
      \begin{cvitems} % description
        \item {A multithreaded program for automating arbitrage trading using the Binance API, written in Rust.}
        \vspace{0.5mm}
		\item {Ensured safe trade execution and profit calculation using message passing between threads.}
		\vspace{0.5mm}
		\item {Received proof of concept funding from York Enterprise.}
		\vspace{0.5mm}
		\item {Deployed on a dedicated bare-metal Debian server with a custom tuned kernel.}
      \end{cvitems}
    }

\begin{comment}
  \cventry
    {}
    {Advantage}
    {}
    {}
    {
      \begin{cvitems}
        \item {A utility for automating options trading written in Python.}
        \vspace{0.5mm}
		\item {Options strategy automaton script that places trades, based on economic indicators, through the Interactive Brokers API.}
		\vspace{0.5mm}
		\item {Heavily utilises statistical Python libraries, most notably: statsmodels, pandas, and numpy.}
      \end{cvitems}
    }
\end{comment}
    
    \cventry
    {}
    {SEC data mining}
    {}
    {}
    {
      \begin{cvitems}
        \item {A program used to perform queries on and derive trading information from an ontology made up of data from SEC reports, written in Python.}
        \vspace{0.5mm}
		\item {Methodology revolves around finding pairs of companies that share links, such as shareholders, prominent employees, or parent company.}
		\vspace{0.5mm}
		\item {Created test suites to investigate the hypothesis that companies sharing these links would be better for pairs trading using short-term, long-term and sliding window historical stock data.}
		\vspace{0.5mm}
		\item {Leveraged statistical Python libraries for analytics, most notably: statsmodels, pandas, and numpy; utilised SPARQL and rdflib to query the ontology.}
      \end{cvitems}
    }
\end{cventries}